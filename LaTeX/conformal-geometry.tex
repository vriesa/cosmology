\chapter{Conformal geometry}


\section{The complex plane}


\section{Holomorphic mappings}


\section{Conformal mappings}


\section{Singularities and infinities}
%%--- Figure: ---------------------------------------------------
\begin{figure}[htp]
\centering
	\includegraphics[scale=1.0]{conformal-diagram-legend}
	\caption{\label{fig:conformal-diagram-legend}
		Legend for reading Penrose diagrams\index{Penrose diagram}.
	}
\end{figure}
%%--- Figure ----------------------------------------------------

%%--- Figure: ---------------------------------------------------
\begin{figure}[htp]
\centering
	\includegraphics[scale=1.0]{FLRW-universes-conformal}
	\caption{\label{fig:FLRW-universes-conformal}
		Penrose diagrams of the FLWR universes.
		Cf. Figure \ref{fig:FLRW-universes}
	}
\end{figure}
%%--- Figure ----------------------------------------------------


\section{Conformal null infinity}

In this section we consider null cones in General Relativity more precisely.
Loosely speaking, the null cones in a space-time represent the sets of all photon trajectories, including closed photon orbits.
A useful technique to study regions of infinity in time or space  is the conformal rescaling of the space-time leading to a compact manifold the finite boundary of which is the set of the spacetime points at infinity. In this way the analytical behavior null cones at infinity can be considered precisely.


One of the most useful areas in general relativity has turned out to
be the study of asymptotic questions, important examples of which are
the definitions of
the total energy-momentum contained in an asymptotically flat
space-time and of gravitational radiation. For this the technique of
conformal rescalings of the space-time $\mathscr{M}$ can be applied, replacing the
original physical metric $\D s$ by a new (“unphysical”) metric
$\D \hat{s}$, which is conformally related to it,
\begin{equation}
\label{eq:ds-conformal}
	\D \hat{s} = \Omega \D s,
\end{equation}
$\Omega$ being a suitably smooth, everywhere positive function defined
on $\mathscr{M}$. The metric tensor $g_{ij}$ and its inverse $g^{ij}$ 
are accordingly rescaled by
\begin{equation}
\label{eq:conformal-metric-components}
	g_{ij} \mapsto \hat{g}_{ij} = \Omega^2 g_{ij},
	\quad
	g^{ij} \mapsto \hat{g}^{ij} = \Omega^{-2} g^{ij}.
\end{equation}
Provided that the asymptotic structure of $\mathscr{M}$ is suitable
and that $\Omega$ is chosen appropriately, it is possible to adjoin
to $\mathscr{M}$ a certain boundary surface, denoted by 
$\mathscr{I}$\index{$\mathscr{I}$}\index{scri}
(pronounced `scri' --- a contraction of `script I'),
in such a way that the “unphysical” metric $\hat{g}_{ij}$ extends
non-degenerately and with some degree of smoothness to these new
points. The function $\Omega$ may also be extended appropriately
but becomes zero on $\scri$. This implies that the physical metric
would have to be infinite on $\scri$, so it cannot be so extended.
Thus, from the point of view of the physical metric, the points
on $\scri$ are infinitely distant from their neighbors.
Physically they represent `points at infinity'.

The adjoining of $\scri$ to a space-time $\mathscr{M}$ provides
us with a smooth manifold with boundary, denoted by 
$\bar{\mathscr{M}}$, i.e.
\begin{equation}
\label{eq:I-M}
	\scri = \partial \mathscr{M},
	\qquad
	\mathscr{M} = \mathrm{int}\ \bar{\mathscr{M}}
\end{equation}
($\partial$ $=$ boundary, int $=$ interior).
The advantage is that the powerful \emph{local} techniques of 
differential geometry can now be employed on $\bar{\mathscr{M}}$
with implications for the asymptotics of $\mathscr{M}$.
Indeed, the very definition of asymptotic flatness in general
relativity can be given in a convenient and coordinate-free
way. Conformal methods are particularly appropriate in
general relativity because many of the important concepts
are conformally invariant. Among these are the massless
free-field equations,\index{massless free-field equation}
the Weyl conformal tensor, null geodesics,
null hypersurfaces, and relativistic causality.
The technique is similar to that used in complex analysis,
where the `point of infinity' is adjoint to the complex
plane to obtain the Riemann sphere\index{Riemann sphere},
and in projective geometry.

\subsection{Minkowski space}
Let us begin by examinig the construction of conformal
infinity for Minkowski space $\mathbb{M}$. It is topologically equivalent to the four-dimensional space, $\mathbb{M} \cong \mathbb{R}^4$.
The physical metric, in spherical polar coordinates, is
\begin{equation}
\label{eq:Minkowski-co}
	\D s^2 = c^2 \D t^2 - \D r^2 
	  - r^2 (\D \theta^2 + \sin^2 \theta \D\varphi^2).
\end{equation}
For convenience we introduce a 
retarded time\index{retarded time} 
parameter $u=ct-r$ and an advanced time\index{advanced time}
parameter $v=ct+r$, to obtain
\begin{equation}
\label{eq:Minkowski-u-v}
	\D s^2 = \D u \D v - \frac{1}{4}\, (v-u)^2 
	 (\D \theta^2 + \sin^2 \theta \D\varphi^2).
\end{equation}
There is much freedom in the choice of a conformal factor $\Omega$.
However, for `asymptotically flat' space-times it turns out
from general considerations (in the context of the 
`peeling theorem'\index{peeling theorem}) that the choice of
$\Omega$ must be such that along any ray it approaches zero,
both in the past and in the future, like the reciprocal
of an affine parameter $\lambda$ on the ray, i.e.\
$\Omega\lambda$ $\to$ constant as $\lambda\to\pm\infty$
along the ray.
Each $u$ $=$ constant hypersurface is a future light cone,
generated by the light rays (null straight lines) for
which $\theta$ and $\varphi$ remain constant.
The coordinate $v$ serves as an affine parameter into the
future on each of these radial rays.
Similarly, the coordinate $u$ serves as an affine parameter
into the past on these rays.
Thus we shall require $\Omega v$ $\to$ constant as $v\to \infty$
on $u,\theta,\varphi$ $=$ constant, and
$\Omega u$ $\to$ constant as $u\to -\infty$
on $v,\theta,\varphi$ $=$ constant.
If we wish to keep $\Omega$ smooth over the finite parts of
space-time, then the choice
\begin{equation}
\label{eq:Omega}
	\Omega = \frac{2}{\sqrt{(1+u^2)(1+v^2)}}
\end{equation}
can be made (the factor 2 being for later convenience),
and thus, by (\ref{eq:ds-conformal}),
\begin{equation}
\label{eq:conformal-metric-Minkowski-uv}
	\D\hat s^2 = \frac{4 \D u\, \D v}{(1+u^2)(1+v^2)}
	  - \frac{(v-u)^2(\D \theta^2 + \sin^2 \theta \D\varphi^2)}
	  {(1+u^2)(1+v^2)}
\end{equation}
Many other choices of $\Omega$ are possible, of course,
but this one is especially convenient. 

In order that the `points at infinity' may be assigned
finite coordinates, we replace $(u,v)$ by $(p,q)$, where
\begin{equation}
\label{eq:p-q}
	u = \tan p, \quad v = \tan q.
\end{equation}
Then
\begin{equation}
\label{eq:conformal-Minkowski}
	\D\hat s^2 = 4\, \D p \D q
	  - \sin^2(q-p)\,(\D\theta^2 + \sin^2\theta \D \varphi^2). 
\end{equation}
The range of the variables $p$, $q$ is as indicated in Figure
\ref{fig:scri}, in which each point represents a 2-sphere of
radius $\sin(q-p)$.
%%--- Figure: ---------------------------------------------------
\begin{figure}[htp]
\centering
	\includegraphics[scale=0.85]{figScri}
	\caption{\label{fig:scri}
		Penrose diagram\index{Penrose diagram}
		of Minkowski space-time $\mathbb{M}$.
		The region of $(p,q)$-space which corresponds to 
		$\mathbb{M}$ and the future (past) null infinity 
		$\mathscr{I}^+$ 
		($\mathscr{I}^-$), defined by
		$q=\frac{\pi}{2}$ ($p=-\frac{\pi}{2}$). 
		The line $q-p=0$ is an axis of spherical symmetry
		(and so also $q-p=\pi$).
		Each point in the diagram defines a 2-sphere
		of radius $\sin(q-p)$.
		Each line $p$ $=$ constant and $q$ $=$ constant 
		represents a light ray.
	}
\end{figure}
%%--- Figure ----------------------------------------------------
The vertical line $q-p=0$ represents the spatial origin ($r=0$)
and is just a coordinate singularity: the space-time is non-singular
on this line. The sloping lines $p=-\frac{1}{2}\pi$,
for $-\frac{1}{2}\pi<q<\frac{1}{2}\pi$, and $q=\frac{1}{2}\pi$,
for $-\frac{1}{2}\pi<p<\frac{1}{2}\pi$, represent null infinity,
denoted by $\scri^-$\index{$\mathscr{I}^-$}
and $\scri^+$\index{$\mathscr{I}^+$}, respectively,
for Minkowski space (corresponding to $u=-\infty$ and
to $v=\infty$).
There are indicated three exceptional points representing 2-spheres
with vanishing radius $\sin(q-p)$, 
the points $\I^\pm$ given by $p=q=\pm \pi/2$, and the point
 $\I^0$ by $q=-p=\pi/2$. They are considered not to belong to
$\scri^-$ or to $\scri^+$.

Physically, we interpret  $\I^-$ as representing past temporal
infinity, $\scri^-$ as past null infinity,  $\I^0$ as spatial
infinity, $\scri^+$ as future null infinity, and  $\I^+$ as future
temporal infinity. The reason for this terminology becomes
clear when we examine the behavior of straight lines
in Minkowski space (with metric $\D s$). A timelike straight
line acquires a past end-point  $\I^-$ and a future end-point
$\I^+$; a null straight line acquires a past end-point
on $\scri^-$ and a future end-point on $\scri^+$;
a spacelike straight line becomes a closed curve through $\I^0$.
Since rays remain rays after conformal rescalings, the null
straight lines become rays with respect to the
$\D\hat s$ metric, whereas the timelike or spacelike
straight lines are not, in general, geodesics
with respect to $\D\hat s$.

However, when we consider \emph{curved} lines in Minkowski space,
the question of which end-points they acquire is more complicated.
For example, the helix curve given by
$\varphi=ct$, $r=1$, $\theta=\pi/2$, is a null curve, since $\D u = \D v = \D \varphi = c \D t$ and $\D r = \D \theta = 0$ in (\ref{eq:conformal-metric-Minkowski-uv}), but has
a past end-point at  $\I^-$ and a future end-point at  $\I^+$,
since $p = \arctan(ct - 1)$, $q = \arctan (ct + 1)$ and $p$, $q \to \pm \infty$ as $t \to \pm \infty$.
At both points the completed curve is not smooth.
On the other hand, the timelike curve $r=c\sqrt{1+t^2}$,
$\theta=\pi/2$, $\varphi=0$, satisfying $\D r^2 = \frac{c^2 t^2 \D t^2}{1 + t^2}$, i.e., $\D s^2 = \frac{c^2}{t^2 + 1} \D t^2 > 0$ in (\ref{eq:Minkowski-co}), smoothly acquires a past-end point
on $\scri^-$ and a future end-point on $\scri^+$, since $p = \arctan (ct - r) \to 0$ and $q = \arctan (ct + r) \to \pi/2$ as $t \to \infty$, and $p \to -\pi/2$ and $q \to 0$ as $t \to - \infty$. 
The only thing one can prove is that two \emph{causal} curves
have the same past-end point on $\scri^-$ iff they share the
same future, and that they have the same future-end point
iff they have the same past.
In particular, a past-endless causal curve acquires a
past end-point at  $\I^-$ or at $\scri^-$ according as
its future is or is not the whole Minkowski space
(since, e.g., the future of the time axis $t=0$ is the 
whole of $\mathbb{M}$). In exactly the same way, a 
future-endless causal curve reaches  $\I^+$ or a point of
$\scri^+$ according as its past set is or is not the whole
Minkowski space \cite[\S9.1]{Penrose-Rindler-1986}.

The metric (\ref{eq:conformal-Minkowski})
is perfectly regular on these regions at infinity,
$\scri^\pm$, and the points  $\I^\pm$,  $\I^0$.
Indeed, the space-time
and its metric $\D\hat s$ can be extended beyond these regions in
a non-singular fashion.
The vertical line $q-p=\pi$ is again a coordinate singularity,
precisely of the same type as that of $q-p=0$. The entire vertical 
strip $0\leqq q-p<\pi$ may be used to define a space-time
$\mathscr{E} \cong \mathbb{R} \times S^3$, the 
\emph{Einstein static universe}\index{Einstein universe}.
To see this, we choose new coordinates
\begin{equation}
	\tau = p+q, \quad \chi = q-p
\end{equation}
and obtain
\begin{equation}
\label{eq:Einstein-space}
	\D s^2 = \D\tau^2
	 - \left[\D\chi^2 
	   + \sin^2\chi\,(\D\theta^2 + \sin^2\theta \D\varphi) \right].
\end{equation}
The part in the square bracket represents the metric of a unit
3-sphere $S^3$. The portion of $\mathscr{E}$ which is conformal
to the original Minkowski space may be described as that lying
between the light cones of two points  $\I^-$ and  $\I^+$.
This portion wraps around the Einstein space to meet at the
`back' in the single point  $\I^0$.
The situation is illustrated in Fig.~\ref{fig:EinsteinCylinder},
under suppression of two dimensions.
%%--- figure: ---------------------------------------------------
\begin{figure}[htp]
\centering
	\includegraphics[scale=0.25]{EinsteinCylinder}
	\caption{\label{fig:EinsteinCylinder}
		The region of the Einstein cylinder $\mathscr{E}$ which
		corresponds to Minkowski space $\mathbb{M}$.
		Figure taken from \cite{Penrose-Rindler-1986}.
	}
\end{figure}
%%--- figure ----------------------------------------------------
Minkowski 2-space is conformal to the interior of a square,
represented as tilted at 45$^\circ$. This square wraps around
the cylinder which is the two-dimensional version of the
Einstein universe.


\subsection{Infinity in Schwarzschild space-time}
\label{sec:Schwarzschild-conformal-null-infinity}
We examine the conformal infinity of the Schwarzschild solution.
The familiar form of the metric is
\begin{eqnarray}
	\D s^2 & \hspace*{-1.75ex} =  \hspace*{-1.75ex} &
	 \left( 1 - \frac{2M}{r}\right) c^2 \D t^2 
	 {}- \left( 1 - \frac{2M}{r}\right)^{-1} \D r^2
	 %\nonumber \\* &  \hspace*{-1.75ex}  \hspace*{-1.75ex} &
	 {}- r^2 (\D\theta^2 + \sin^2\theta \D\varphi^2). 
\label{Schwarzschild-metric}
\end{eqnarray}
Rather than attempting to obtain $\scri^+$ and $\scri^-$ simultaneously,
as was done for Minkowski space, it is more appropriate to introduce
a retarded time coordinate 
\begin{equation}
\label{Schwarzschild-u}
	u = ct - r - 2M \ln (r-2M)
\end{equation}
and an advanced time coordinate
\begin{equation}
\label{Schwarzschild-v}
	v = ct + r + 2M \ln (r-2M)
\end{equation}
\emph{separately}. In the first case the metric form becomes
\begin{eqnarray}
	\D s^2 & \hspace*{-1.75ex} =  \hspace*{-1.75ex} &
	 \left( 1 - \frac{2M}{r}\right) \D u^2 
	 {}+ 2 \D u \D r
	 %\nonumber \\* &  \hspace*{-1.75ex}  \hspace*{-1.75ex} &
	 {}- r^2 (\D\theta^2 + \sin^2\theta \D\varphi^2),
\label{Schwarzschild-metric-u-r}
\end{eqnarray}
and in the second one,
\begin{eqnarray}
	\D s^2 & \hspace*{-1.75ex} =  \hspace*{-1.75ex} &
	 \left( 1 - \frac{2M}{r}\right) \D v^2 
	 {}- 2 \D v \D r
	 %\nonumber \\* &  \hspace*{-1.75ex}  \hspace*{-1.75ex} &
	 {}- r^2 (\D\theta^2 + \sin^2\theta \D\varphi^2). 
\label{Schwarzschild-metric-v-r}
\end{eqnarray}
In each case we can choose $\Omega = r^{-1} = w$, say. Then
the “unphysical” metric is
\begin{eqnarray}
	\D \hat s^2 & \hspace*{-1.75ex} =  \hspace*{-1.75ex} &
	 \Omega^2 \D s^2 =
	 \left( w^2 - 2Mw^3\right) \D u^2 
	 {}- 2 \D u \D w
	 %\nonumber \\* &  \hspace*{-1.75ex}  \hspace*{-1.75ex} &
	 {}- \D\theta^2 + \sin^2\theta \D\varphi^2 
\label{Schwarzschild-metric-u-w}
\end{eqnarray}
in the first case, and
\begin{eqnarray}
	\D \hat s^2 & \hspace*{-1.75ex} =  \hspace*{-1.75ex} &
	 \left( w^2 - 2Mw^3\right) \D v^2 
	 {}+ 2 \D v \D w
	 %\nonumber \\* &  \hspace*{-1.75ex}  \hspace*{-1.75ex} &
	 {}- \D\theta^2 + \sin^2\theta \D\varphi^2 
\label{Schwarzschild-metric-v-w}
\end{eqnarray}
in the second one. The metrics (\ref{Schwarzschild-metric-u-w}) and
(\ref{Schwarzschild-metric-v-w}) are manifestly regular and analytic
on their respective hypersurfaces $w=0$. (Clearly the determinants
are non-zero at $w=0$.) The physical space-time is given when
$w>0$ in (\ref{Schwarzschild-metric-u-w}) and we can extend the manifold
to include the boundary hypersurface $\scri^+$, given when $w=0$.
Similarly, in (\ref{Schwarzschild-metric-v-w}) the physical space-time
corresponds to $w>0$ and can be extended to include $\scri^-$,
given when $w=0$. In fact, we could extend the space-time
across $w=0$ to negative values of $w$, but this will not be
done here. Only the boundary $\scri$ $=$ $\scri^-$ $\cup$ $\scri^+$
will be adjoined to the space-time.

There is a difficulty if we try to identify $\scri^-$ with $\scri^+$.
If we do extend the region of definition of 
(\ref{Schwarzschild-metric-u-w}) to include negative values
of $w$, and then make the replacement $w \mapsto -w$, we see that
the metric has the form (\ref{Schwarzschild-metric-v-w})
with $u$ in place of $v$, but with a negative mass $-M$ in place of $M$.
Thus, the extension across $\scri$ involves a reversal of the sign
of the mass. In fact, the derivative at $\scri$ of the conformal
curvature contains the information of the mass. Therefore, if we
attempt to identify $\scri^+$ with $\scri^-$, and want the \emph{same}
sign of the non-zero mass $M$ to occur on the two sides, then there
must be a discontinuity in the derivative of the curvature across
$\scri$, so that the metric must fail to be $C^3$ at $\scri$.

Accepting, then, that is is not reasonable to identify $\scri^+$ with
$\scri^-$, we are led to a picture closely resembling that obtained
for Minkowski space. The only essential difference occurs with the
points $i^-$, $i^0$, $i^+$. It turns out that whenever mass is present,
the point $i^0$, and normally also $i^\pm$, must, if adjoined to the
manifold, be singular for the conformal geometry. It is therefore
reasonable not to attempt to include these points, in the general
case, as part of the conformal infinity. (Even in Minkowski space
the boundary surface at $i^0$, $i^\pm$ is not smooth.)
The picture, then, is as indicated in Fig.~\ref{fig:scriBlackHole}.
%%--- figure: ---------------------------------------------------
\begin{figure}[htp]
\centering
	\includegraphics[scale=1.4]{scriBlackHole}
	\vspace*{-1ex}
	\caption{\label{fig:scriBlackHole}
		Null infinity for the Schwarzschild space-time. Note that
		$w=0$ corresponds both to $\scri^+$ and to $\scri^-$.
%		The points $i^\pm$, $i^0$ are singular (divergent Weyl
%		curvature) and have been deleted. This picture serves as
%		a model for asymptotically flat spaces generally.
		Figure taken from \cite{Penrose-Rindler-1986}.
	}
\end{figure}
%%--- figure ----------------------------------------------------
We have two disjoint boundary hypersurfaces $\scri^-$ and $\scri^+$
each of which is a “cylinder” with topology $S^2 \times \mathbb{R}$.
It is clear from (\ref{Schwarzschild-metric-u-w}) and 
(\ref{Schwarzschild-metric-v-w}) that each of $\scri^\pm$ is a 
\emph{null} hypersurface, the induced metric at $w=0$ being degenerate.
These null hypersurfaces are generated by rays, given by 
$\theta$, $\varphi$ $=$ constant, $w=0$, whose tangents are
\emph{normals} to the hypersurfaces. These rays may be taken to be the
“$\mathbb{R}$s” of the topological product $S^2 \times \mathbb{R}$.

The picture in Fig.~\ref{fig:scriBlackHole} serves as a model for asymptotically flat spaces generally.
